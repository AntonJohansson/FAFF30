% Laborationsmall.tex
\documentclass[a4paper]{article}

\usepackage[swedish]{babel}
\usepackage[utf8x]{inputenc}

\usepackage{multicol}
\usepackage[vmargin=3cm,hmargin=2cm]{geometry}
\usepackage{parskip}
\usepackage[runin]{abstract}
\renewcommand{\abstitleskip}{0mm}

\usepackage{hyperref}
\usepackage{amsmath}
\usepackage{lmodern}
\usepackage[T1]{fontenc}
\usepackage{pgfplots}
\pgfplotsset{compat=1.13}
\usetikzlibrary{decorations.pathreplacing}
\usetikzlibrary{arrows.meta}
\usepackage{placeins}
\usepackage{siunitx}
\usepackage[numbered,framed]{matlab-prettifier}

\lstset{
	style              = Matlab-editor,
	basicstyle         = \mlttfamily,
	mlshowsectionrules = true,
}

\addto\extrasswedish{%
	\def\equationautorefname{Ekvation}%
	\def\figureautorefname{Figur}%
	\def\tableautorefname{Tabell}%
	\def\sectionautorefname{Rubrik}%
	\def\subsectionautorefname{Underrubrik}%
	\def\pageautorefname{Sida}%
}

\usepackage{graphicx}
\usepackage{ccaption}
\captionnamefont{\it}
\captiontitlefont{\it}

% Hack för att få komma istället för punkt i matematiska uttryck
% $3.141592$ blir 3,141592
% Om man använder komma direkt får man ett litet oönskat mellanrum:
% $3,141592$ blir 3, 141592
\DeclareMathSymbol{,}{\mathpunct}{letters}{"3B}
\DeclareMathSymbol{.}{\mathord}{letters}{"3B}
\DeclareMathSymbol{\decimal}{\mathord}{letters}{"3A}

% Kommando för att få icke-kursiva enheter i matematiska uttryck
% $10\unit{km}$ blir 10 km
\newcommand{\unit}[1]{\ensuremath{\,\mathrm{#1}}}

\usepackage{lastpage}
\usepackage{fancyhdr}
\pagestyle{fancy}
\fancyhf[C]{\thepage}
\fancyhead[C]{Våglära och optik, FAFF30}
\fancyhead[R,L]{}
\fancypagestyle{plain}{
  \fancyhead{}
}
%\setcounter{secnumdepth}{-1} % Stänger av numrering.

\title{Laborationen ”Ljusets diffraktion”}
\author{Johan Boström\\Anton Johansson\\Lunds Universitet}

\makeatletter
%\renewcommand{\section}{\@startsection
%{section}%                   % the name
%{1}%                         % the level
%{0mm}%                       % the indent
%{-\baselineskip}%            % the before skip
%{0.5mm}%          % the after skip
%{\normalfont\bfseries}} % the style
%\renewcommand{\sectionmark}[1]{ }
%\renewcommand{\thesection}{}

\renewcommand*\maketitle{
  {
    \begin{center}
      {\huge\bfseries \@title}\\
      \vspace{5mm}
      {\large \@author}
    \end{center}
    \vspace{2mm}
  }
}
\makeatother

\begin{document}
\maketitle

\renewcommand{\abstractname}{Abstract} % Om vi vill ha titeln till abstracten på engelska.

\begin{abstract}
  %Kort beskrivning  av resultaten (5-10  rader). Det ska inte  vara en innehållsbeskrivning (först gör vi det, sen använder vi den metoden, och så jämför vi det med de där tidigare kända resultaten, etc) utan vara   koncentrerat  till   ”resultatet”,   vad   man  kommer fram till. Sammanfattningen är uppsatsens löpsedel. Den ska vara kort och kärnfull och locka läsare genom att  effektivt göra klart vad det är man uppnår genom att läsa rapporten.
  
\end{abstract}

\vspace{2mm}

\hspace{-3mm}
\begin{tabular}{ll}
Laborationen genomförd: &	2017-05-04 \\
%Rapporten kamratgranskad: &	2013-xx-xx \\
Lämnad till handledare: &	2017-xx-xx \\
\end{tabular}

\vspace{3mm}

\section{Inledning}
  %Preliminär beskrivning av uppgiften  i lättfattliga termer. Bakgrunden till att man intresserar sig för detta problem, denna uppgift. Hur den ingår i ett större sammanhang.

%  En detaljerad beskrivning av vad det hela går ut på, och vad exakt din uppgift   i    sammanhanget   är.    Utrustningsförutsättningar.   Det
%  ursprungligen avsedda målet. Även den i ämnet oinsatte ska ha en ärlig
%  chans att förstå de stora dragen.
%
%  Beskrivning av hur rapporten är uppbyggd, för att göra det möjligt för
%  läsaren att förstå vad som  pågår, ge läsaren rätta förväntningar, och
%  för  att  underlätta direktåtkomst  av  för  den individuelle  läsaren
%  särskilt intressanta delar.

\section{Teori}

\subsection{Diffraktion}

\pgfmathsetmacro{\numSources}{4}
\pgfmathsetmacro{\numWaves}{10}
\pgfmathsetmacro{\numIncoming}{8}
\pgfmathsetmacro{\wavesDist}{0.25}

När en våg passerar en spalt eller ett annat objekt som blockerar ljuset i vissa områden kommer den enligt Huygens princip att uppföra sig som om det funnits en mängd punktkällor utspridda över hela ytan där objektet är genomskinligt. Vågorna som bildas från dessa kommer därefter att interferera med varandra. Detta demonstreras i \autoref{fig:singleInterference} med kollimerat infallande ljus approximerat med endast \numSources~punktkällor.

\begin{figure}[ht]
	\centering
	\begin{tikzpicture}
	\draw[line width=0.75mm] (0, {1 + \wavesDist*\numWaves}) -- (0, 0.5);
	\draw (0, 0.5) -- (0, -0.5);
	\draw[line width=0.75mm] (0, -0.5) -- (0, {-1 - \wavesDist*\numWaves});
	\foreach \incoming in {1,...,\numIncoming}
	{
		\draw ({-\incoming*\wavesDist}, {-0.5 - \wavesDist*\numWaves}) -- ({-\incoming*\wavesDist}, {0.5 + \wavesDist*\numWaves});
	}
	\foreach \source in {1,...,\numSources} % For each source
	{
		\node[circle,fill=black,inner sep=0,minimum size=5] at (0, {(\source-1)/(\numSources-1) - 0.5}) {}; % Draw the source
		\foreach \waveIdx in {1,...,\numWaves}
		{ % and numWaves waves spaced with wavesDist
			\draw (0, {(\source-1)/(\numSources-1) - 0.5 - \wavesDist*\waveIdx}) arc [radius={\wavesDist*\waveIdx}, start angle=-90, end angle=90];
		}
	}
	\end{tikzpicture}
	\caption{Demonstration av diffraktion i enkelspalt då vågorna infaller från vänster. Där vågorna överlappar interfererar de konstruktivt, medan det är svårare att se när de interfererar destruktivt. Förutom huvudmaximat rakt åt höger syns även ett ytterliggare maximum i vardera riktningen. I en riktig spalt blir det oändligt många punktkällor som måste adderas ihop i en integral.}
	\label{fig:singleInterference}
\end{figure}

För att summera bidragen från de oändligt många punktkällorna används Fresnel-Kirchhoffs integral \cite[p.~330]{pearsonIntroOpt}

\begin{equation}
	E_P = \frac{-i k E_S}{2\pi} e^{-i \omega t} \iint\limits_{\mathrm{Objektet}} {F(\theta)\frac{e^{i k (r + r')}}{r r'} d A}
	\label{eq:frKirInt}
\end{equation}

där $E_P$ är det elektriska fältet på skärmen med avstånder $r$ från den infinitesimala arean $d A$ på objektet. På samma sätt är $E_S$ det elektriska fältet av punktkällan och $r'$ dess avstånd till $d A$. $k$ är vågtalet, $\omega$ är vinkelfrekvensen och $F(\theta) = \frac{1 + \cos\theta}{2}$ är en obliquity-faktor som är nära ett då vågen passerar rakt igenom objektet men avtar för högra vinklar.

\subsubsection{Fraunhoferdiffraktion}

När både källan och skärmen är långt borta från objektet varierar både $r$ och $r'$ i \eqref{eq:frKirInt} mycket lite över objektets area och kan flyttas ut ur integralen. Samtidigt blir obliquity-faktorn nästan konstant då $\theta$ är nära noll och även den kan tas ut ur integralen. Integralen kan då förenklas till \cite[p.~331]{pearsonIntroOpt}

\begin{equation}
	E_P = C_0 e^{-i \omega t} \iint\limits_{\mathrm{Objektet}} {e^{i k r} dA}
	\label{eq:fraunhofer}
\end{equation}

där $C_0$ är en konstant. Om positionen på skärmen betecknas med ortsvektorn $\boldsymbol{p}$ och positionen på objektet med $\boldsymbol{p'}$ blir

\begin{equation}
	r = \sqrt{(\boldsymbol{p} - \boldsymbol{p'})\cdot(\boldsymbol{p} - \boldsymbol{p'})} = \sqrt{\boldsymbol{p}\cdot\boldsymbol{p} - 2 \boldsymbol{p}\cdot\boldsymbol{p'} + \boldsymbol{p'}\cdot\boldsymbol{p'}}\text{.}
\end{equation}

Man kan nu bryta ut $|\boldsymbol{p}|$

\begin{equation}
r = |\boldsymbol{p}|\sqrt{1 - \frac{2\boldsymbol{p}\cdot\boldsymbol{p'}}{|\boldsymbol{p}|^2} + \left(\frac{|\boldsymbol{p'}|}{|\boldsymbol{p}|}\right)^2}
\end{equation}

och då $|\boldsymbol{p'}|\ll|\boldsymbol{p}|$ blir $\left(\frac{|\boldsymbol{p'}|}{|\boldsymbol{p}|}\right)^2\approx 0$ och man kan utveckla kvadratroten till

\begin{equation}
	r = |\boldsymbol{p}|\left( 1 - \frac{\boldsymbol{p}\cdot\boldsymbol{p'}}{|\boldsymbol{p}|^2} \right) = |\boldsymbol{p}| - \left(\frac{\boldsymbol{p}}{|\boldsymbol{p}|}\right)\cdot\boldsymbol{p'}\text{.}
\end{equation}

Här är $|\boldsymbol{p}|$ konstant i integralen i \eqref{eq:fraunhofer} och kan absorberas i konstanten. Med beteckningen $\boldsymbol{k} = k \frac{\boldsymbol{p}}{|\boldsymbol{p}|}$ kan \eqref{eq:fraunhofer} skrivas

\begin{equation}
	E_P = C_1 e^{-i \omega t} \iint\limits_{\mathrm{Objektet}} {e^{-i \boldsymbol{k}\cdot\boldsymbol{p'}} dA}\text{.}
\end{equation}

Nu är integralen exakt definitionen av en tvådimensionell Fouriertransform av en funktion som är ett där objektet släpper igenom ljus och noll där den inte gör det.

\subsubsection{Fresneldiffraktion}

Då man inte kan göra de approximationer som leder fram till Fraunhoferdiffraktion måste man lösa \eqref{eq:frKirInt} direkt, vilket bara går analytiskt för ett fåtal specialfall. Två sådana specialfall behandlas i \cite{pearsonIntroOpt}: objekt med cirkulär samt rektangulär symmetri. Ytterliggare beräknas integralen för en rektangulär öppning då variationerna av obliquity-faktorn och produkten $r r'$ försummas.

% Cirkulära vågfronter, analogi med lök?

\section{Metod}
  % De  olika  stegen  i  uppgiftens genomförande.  Till  exempel  val  av
  % algoritmer,  programspråk och  annan programvara,  undersökningsmetod,
  % statistiska  metoder. Där  valmöjligheter finns,  diskutera de  gjorda
  % valen.
  
  En laserstråle linjerades och kollimerades och riktades mot en skärm. Bakom skärmen placerades en kamera som kopplades in till ett program på en dator. En hållare som till exempel kan innehålla olika spaltsystem placerades på lämpligt avstånd från skärmen så att alla mönster blev av rimliga storlekar.
  För att mäta detta avstånd användes en dubbelspalt med kända mått och intensitetsfördelningen passades in på en teoretiskt beräknad kurva i programmet. Sedan placerades i tur och ordning en enkelspalt, en tråd, ett flerspaltsystem och en ställbar spalt på samma avstånd från skärmen och intensitetsfördelningarna mättes. Intensitetsfördelningarna passades sedan in över teoretiska intensitetsfördelningar så att relevanta parametrar kunde beräknas.
  
  Sedan togs utrustningen som kollimerade laserstrålen bort och den divergerande strålen läts belysa en cirkel, en ställbar spalt samt en egg. Diffraktionsmöstren fotograferades och för den ställbara spalten filmades förändringarna då spaltbredden förändrades.
  
  \subsection{Datorsimulering}
  
  För att kunna jämföra med vad teorin förutsäger skrevs ett antal skript i matlab som skulle simulera Fraunhofer- och Fresneldiffraktion. Fraunhoferdiffraktionen implementerades med hjälp av matlabs inbyggda FFT algoritm \lstinline{fft2} och Fresneldiffraktion implementerades för specialfallet med en egg genom funktionerna  \lstinline{fresnelc} och \lstinline{fresnels}.
  
  \begin{figure}[ht] % Exempel på matlab kod.
  \centering
  \begin{lstlisting}
function a = f(b)
	a = b .^ 2; % Matlab code example.
end
  \end{lstlisting}
  \end{figure}

\section{Resultat}

\section{Diskussion}
  %Är resultaten rimliga? Vad hade kunnat göras annorlunda?

\section{Slutsats}
  %En   sammanfattning  där   man  till   skillnad  från   den  inledande
  %sammanfattningen förutsätter  att läsaren har läst  rapporten, samt de
  %slutsatser man kan dra av det gjorda arbetet.
 
 \bibliography{bibliography}{}
 \bibliographystyle{plain}

\end{document}
